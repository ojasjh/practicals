\documentclass{article}
\usepackage{amsmath}
\usepackage[margin=1in]{geometry}

\title{Equations}
\author{Prof. Naveen Kumar$^1$, Dr. Neeraj Kumar Sharma$^2$, and Sakeena Shahid$^3$\\
\small $^1$Department of Computer Science, University of Delhi\\
\small $^2$Ram Lal Anand College, University of Delhi\\
\small $^3$SGTB Khalsa College, University of Delhi}
\date{November 15, 2022}

\begin{document}

\maketitle

\section{Maxwell's Equations}

``Maxwell's equations'' are named for James Clark Maxwell and are as follows:

\begin{align}
\vec{\nabla} \cdot \vec{E} &= \frac{\rho}{\epsilon_0} & \text{Gauss's Law}\label{eq:gauss}\\
\vec{\nabla} \cdot \vec{B} &= 0 & \text{Gauss's Law for Magnetism}\label{eq:gaussmag}\\
\vec{\nabla} \times \vec{E} &= -\frac{\partial\vec{B}}{\partial t} & \text{Faraday's Law of Induction}\label{eq:faraday}\\
\vec{\nabla} \times \vec{B} &= \mu_0\left(\epsilon_0\frac{\partial\vec{E}}{\partial t} + \vec{J}\right) & \text{Ampere's Circuital Law}\label{eq:ampere}
\end{align}

Equations 1, 2, 3, and 4 are some of the most important in Physics.

\section{Matrix Equations}

\begin{equation}
\begin{pmatrix}
a_{11} & a_{12} & \cdots & a_{1n} \\
a_{21} & a_{22} & \cdots & a_{2n} \\
\vdots & \vdots & \ddots & \vdots \\
a_{n1} & a_{n2} & \cdots & a_{nn}
\end{pmatrix}
\begin{pmatrix}
v_1 \\
v_2 \\
\vdots \\
v_n
\end{pmatrix} =
\begin{pmatrix}
w_1 \\
w_2 \\
\vdots \\
w_n
\end{pmatrix}
\end{equation}

\end{document}
